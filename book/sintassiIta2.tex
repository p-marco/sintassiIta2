\PassOptionsToPackage{unicode=true}{hyperref} % options for packages loaded elsewhere
\PassOptionsToPackage{hyphens}{url}
%
\documentclass[a4paper,twoside,11pt,chapterprefix=false,bibliography=totocnumbered,listof=flat]{scrbook}
\usepackage{lmodern}
\usepackage{amssymb,amsmath}
\usepackage{ifxetex,ifluatex}
\usepackage{fixltx2e} % provides \textsubscript
\ifnum 0\ifxetex 1\fi\ifluatex 1\fi=0 % if pdftex
  \usepackage[T1]{fontenc}
  \usepackage[utf8]{inputenc}
  \usepackage{textcomp} % provides euro and other symbols
\else % if luatex or xelatex
  \usepackage{unicode-math}
  \defaultfontfeatures{Ligatures=TeX,Scale=MatchLowercase}
\fi
% use upquote if available, for straight quotes in verbatim environments
\IfFileExists{upquote.sty}{\usepackage{upquote}}{}
% use microtype if available
\IfFileExists{microtype.sty}{%
\usepackage[]{microtype}
\UseMicrotypeSet[protrusion]{basicmath} % disable protrusion for tt fonts
}{}
\IfFileExists{parskip.sty}{%
\usepackage{parskip}
}{% else
\setlength{\parindent}{0pt}
\setlength{\parskip}{6pt plus 2pt minus 1pt}
}
\usepackage{hyperref}
\hypersetup{
            pdftitle={Sintassi Italiana 2},
            pdfauthor={Marco Petolicchio},
            pdfborder={0 0 0},
            breaklinks=true}
\urlstyle{same}  % don't use monospace font for urls
\usepackage{longtable,booktabs}
% Fix footnotes in tables (requires footnote package)
\IfFileExists{footnote.sty}{\usepackage{footnote}\makesavenoteenv{longtable}}{}
\setlength{\emergencystretch}{3em}  % prevent overfull lines
\providecommand{\tightlist}{%
  \setlength{\itemsep}{0pt}\setlength{\parskip}{0pt}}
\setcounter{secnumdepth}{5}
% Redefines (sub)paragraphs to behave more like sections
\ifx\paragraph\undefined\else
\let\oldparagraph\paragraph
\renewcommand{\paragraph}[1]{\oldparagraph{#1}\mbox{}}
\fi
\ifx\subparagraph\undefined\else
\let\oldsubparagraph\subparagraph
\renewcommand{\subparagraph}[1]{\oldsubparagraph{#1}\mbox{}}
\fi

% set default figure placement to htbp
\makeatletter
\def\fps@figure{htbp}
\makeatother

\usepackage[a4paper,margin=4cm, bindingoffset=1cm, heightrounded, headsep=2em, footskip=11mm, vmarginratio=1:1]{geometry}
\makeatletter
\DeclareOldFontCommand{\rm}{\normalfont\rmfamily}{\mathrm}
\DeclareOldFontCommand{\sf}{\normalfont\sffamily}{\mathsf}
\DeclareOldFontCommand{\tt}{\normalfont\ttfamily}{\mathtt}
\DeclareOldFontCommand{\bf}{\normalfont\bfseries}{\mathbf}
\DeclareOldFontCommand{\it}{\normalfont\itshape}{\mathit}
\DeclareOldFontCommand{\sl}{\normalfont\slshape}{\@nomath\sl}
\DeclareOldFontCommand{\sc}{\normalfont\scshape}{\@nomath\sc}
\makeatother




\usepackage{fontspec}

\usepackage{ifluatex}

\usepackage{microtype}



\setmainfont[Numbers=Lowercase]{IBMPlexSerif}
\setsansfont[Numbers=Lowercase]{IBMPlexSans}
\setmonofont[Numbers=Lowercase]{IBMPlexMono}


\usepackage[italian]{babel}

\usepackage[all]{nowidow}

\usepackage[usenames, dvipsnames]{color}
\definecolor{upol-dGrey}{rgb}{0.36470588235,0.36862745098,0.37647058823}
\definecolor{upol-lGrey}{rgb}{0.8,0.8,0.8}
\definecolor{upol-brandBlue}{rgb}{0,0.43529411764,0.67843137254}


\usepackage{xcolor}
\usepackage{graphicx}
\definecolor{titlepagecolor}{cmyk}{1,.38,0,.15} %C100 M38 Y0 K15
\definecolor{namecolor}{cmyk}{0, 0, 0, .0980} 
\usepackage{textcase}


\usepackage{setspace}

\makeatletter\let\Title\@title\makeatother
%\makeatletter\let\Author\@author\makeatother



\usepackage{sectsty}
\allsectionsfont{\color{upol-dGrey}\sffamily}
\chapterfont{\color{upol-dGrey}\raggedleft\thispagestyle{empty}}





\usepackage{floatrow}
\floatsetup[table]{font=sf}
\floatsetup[figure]{font=sf}
\floatsetup[tikzpicture]{font=sf}

\usepackage[font={color=upol-dGrey}, labelfont={color=upol-dGrey}]{caption}



\makeatletter
\def\verbatim@font{\linespread{1}\footnotesize\ttfamily}
\makeatother



\makeatletter
\renewenvironment{figure}[1][\fps@figure]{
  \edef\@tempa{\noexpand\@float{figure}[#1]} 
  \@tempa
  \sffamily
}{
  \end@float
}
\renewenvironment{table}[1][\fps@table]{
  \edef\@tempa{\noexpand\@float{table}[#1]} 
  \@tempa
  \sffamily
  \footnotesize
}{
  \end@float
}
\makeatother

\usepackage{tabularx}
\usepackage{amsfonts}
\usepackage{booktabs}
\usepackage{siunitx}
\usepackage{fancyhdr}

\usepackage{lipsum, kantlipsum} % just for testing

\pagestyle{fancy}
\fancyhf{}
\fancyhead[LE,RO]{\thepage}
\fancyhead[RE]{\footnotesize\nouppercase{\leftmark}}
\fancyhead[LO]{\footnotesize\nouppercase{\rightmark}}
\setlength{\headheight}{14.5pt} % as requested by fan
\renewcommand{\headrulewidth}{0pt}

\renewcommand{\chaptermark}[1]{\markboth{\thechapter \ . \  #1}{}}
\renewcommand{\sectionmark}[1]{\markright{\thesection \ \ #1}{}}





%\setcounter{secnumdepth}{5}
%\setsecnumdepth{subsubsection}
%\maxtocdepth{subsubsection}


\setlength{\skip\footins}{3em}
\renewcommand\footnoterule{{\hrule height 0pt}} % a long blue line



\usepackage{colortbl}
\arrayrulecolor{gray}







\usepackage{booktabs}

\usepackage{pdfpages}






%Options: Sonny, Lenny, Glenn, Conny, Rejne, Bjarne, Bjornstrup
\usepackage[Bjornstrup]{fncychap}
%\renewcommand{\CNoV}{\raggedleft\sffamily\selectfont\HUGE}
  \ChNumVar{\Huge\sffamily\selectfont}
\renewcommand{\DOCH}{%
   \settowidth{\py}{\CNoV\thechapter}
  \addtolength{\py}{1em}      % Amount of space by which the
%                                % number is shifted right
   \fboxsep=0pt%
   \colorbox[gray]{.85}{\rule{0pt}{50pt}\parbox[b]{\textwidth}{\hfill}}%
   \kern-\py\raise20pt%
   \hbox{\color{gray}\CNoV\thechapter}\\%
}

\makeatletter
\renewcommand*{\@makechapterhead}[1]{%
  \vspace*{0\p@}%
  {\parindent \z@ \raggedright \normalfont
    \ifnum \c@secnumdepth >\m@ne
      \if@mainmatter%%%%% Fix for frontmatter, mainmatter, and backmatter 040920
        \DOCH
      \fi
    \fi
    \interlinepenalty\@M
    \if@mainmatter%%%%% Fix for frontmatter, mainmatter, and backmatter 060424
      \DOTI{#1}%
    \else%
      \DOTIS{#1}%
    \fi
  }}
% For the case \chapter*:
\renewcommand*{\@makeschapterhead}[1]{%
  \vspace*{10\p@}%
  {\parindent \z@ \raggedright
    \normalfont
    \interlinepenalty\@M
    \DOTIS{#1}
    \vskip 0\p@
  }}
\makeatother




\renewcommand*\chapterpagestyle{empty}




\usepackage{tikz,tikz-qtree}


\tikzstyle{every picture}+=[font=\sffamily]

\usepackage{listings}
\lstset{ %Formatting for code in appendix
	backgroundcolor = \color{gray},
	language=Python,
	basicstyle=\singlespacing\footnotesize\ttfamily\color{white},
	numbers=left,
	stepnumber=1,
	showstringspaces=true,
	tabsize=2,
	breaklines=true,
	breakatwhitespace=false,
	xleftmargin=3em,framexleftmargin=3em, numberstyle=\ttfamily,
}

\usepackage{datetime}

\let\oldmaketitle\maketitle
\AtBeginDocument{\let\maketitle\relax}




\usepackage{catchfilebetweentags}

\usepackage[autostyle]{csquotes}  
\usepackage{enumerate}


\makeatletter
\newcommand\HUGE{\@setfontsize\Huge{36}{48}} 
\makeatother

\usepackage{tikz}
\usepackage{pgfplots}


\usepackage{gb4e,cgloss}
\noautomath

\let\eachwordone=\it

\frontmatter
\usepackage[]{natbib}
\bibliographystyle{apa}

\title{Sintassi Italiana 2}
\providecommand{\subtitle}[1]{}
\subtitle{Dispense per gli studenti del corso KRI/SYNT2}
\author{Marco Petolicchio}
\date{2018-12-22}

\begin{document}
\maketitle

\makeatletter
\begin{titlepage}
\color{namecolor}

\pagecolor{titlepagecolor}

\noindent
\makebox[0pt][l]{\rule{1.4\textwidth}{1pt}}
\par

\noindent
\textbf{\sffamily{Palacký University}} \textcolor{namecolor}{\sffamily{ Olomouc}}\par \vskip\baselineskip 

\noindent\sffamily{Dipartimento di Lingue Romanze, Facoltà di Filosofia} \par\vskip\baselineskip

\vspace{2em}

\begin{flushleft}

\noindent
{\HUGE \sf \@title  \par}\vskip\baselineskip

\noindent
{\Large \sf \@subtitle \par} \vskip\baselineskip
\end{flushleft}

\noindent
{\large{\sffamily\MakeTextUppercase{\@author}}\par} 
\vspace{4em}
\par \vskip\baselineskip
\vspace{3em}

 {\footnotesize   \ttfamily{draft : : \today   : :  \currenttime} \par}

\color{black}

\end{titlepage}
\makeatother
\nopagecolor% Use this to restore the color pages to white
\pagecolor{white}

{
\setcounter{tocdepth}{1}
\tableofcontents
}
\listoftables
\listoffigures
\hypertarget{introduzione}{%
\chapter*{Introduzione}\label{introduzione}}
\addcontentsline{toc}{chapter}{Introduzione}

Questa dispensa nasce come materiale di studio per l'esame di Sintassi
Italiana 2 per gli studenti triennali dell'Università Palacky di
Olomouc, pensata in maniera specifica per studenti non madrelingua. Si
fa riferimento a nozioni \emph{tradizionali} della linguistica e degli
studi sintattici rimandando, laddove si è ritenuto più pertinente, a
degli studi più recenti in maniera da poter stimolare ulteriormente lo
studente.

Per qualsiasi informazione è possibile scrivere direttamente all'autore
all'indirizzo \texttt{marco.petolicchio01@upol.cz}.

Quest'opera è rilasciata con licenza Creative Commons BY 4.0. Il codice
sorgente è disponibile all'indirizzo
\texttt{http://github.com/p-marco/sintassiIta2} e le versioni del
progetto sono rilasciate in DOI attraverso la piattaforma Zenodo
(\url{DOI:10.5281/zenodo.2355707}).

\mainmatter

\part*{Parte I. \\\ Questioni preliminari}

\hypertarget{le-parti-del-discorso}{%
\chapter{Le parti del discorso}\label{le-parti-del-discorso}}

Le parole di una lingua vengono divise all'interno di categorie
grammaticali. In italiano --una lingua flessiva come buona parte delle
lingue indoeuropee \citep{graffiScalise2009}-- queste suddivisioni
avvengono per criteri di natura sintattica, ovvero la posizione ed il
ruolo delle parole all'interno della frase. Tradizionalmente possiamo
riconoscere 9 diverse \textbf{parti del discorso} \citep{salvi2013}, tra
cui possiamo operare una ulteriore suddivisione: quelle (parti)
\emph{variabili} e quelle \emph{invariabili}.

\hypertarget{le-parti-variabili}{%
\section{Le parti variabili}\label{le-parti-variabili}}

In italiano si definiscono parti \textbf{variabili} del discorso quelle
che hanno la possibilità di modificarsi sulla base di alcuni
\emph{tratti} o \emph{categorie grammaticali} \citep[ Cap.9]{simone1995}
come il Genere, il Numero, la Persona, il Caso, il Tempo, l'Aspetto, il
Modo ecc..

\hypertarget{aggettivo}{%
\subsection{Aggettivo}\label{aggettivo}}

L'aggettivo è un \emph{modificatore} di altri elementi del discorso,
soprattutto del sostantivo, con cui instaura un rapporto sintattico che
si manifesta, nella maggior parte dei casi, nella concordanza
grammaticale (Brutto \emph{stamani il} tempo \emph{e ancora più}
pestifero \emph{il} Tempo \citep{montale-satura}, \emph{Le} lasagne
scaldate \emph{nel micro che da solo mi sento} cattivo
\citep{fibra2017}).

Tradizionalmente possiamo suddividere la classe di aggettivi in due
categorie:

\begin{itemize}
\tightlist
\item
  Determinativi:

  \begin{itemize}
  \tightlist
  \item
    Possessivi (\emph{mia, vostre, suo})
  \item
    Numerali:

    \begin{itemize}
    \tightlist
    \item
      Cardinali (\emph{due, trentatré})
    \item
      Ordinali (\emph{primo, quarantatreesimo})
    \end{itemize}
  \item
    Dimostrativi (\emph{questo, quello})
  \item
    Indefiniti (\emph{alcuni, tutti, nessuna})
  \item
    Interrogativi ed esclamativi (\emph{quale?, quanti?, quale gioia!,
    ma che onore!})
  \end{itemize}
\item
  Qualificativi (\emph{forte, grande, bello, rettangolare, goloso,
  verde, vecchio})
\end{itemize}

I determinativi esprimono alcune funzioni della referenza (per esempio
il possesso), mentre i qualificativi esprimono dei caratteri quali il
colore, la forma, l'aspetto, le qualità. Quella dei determinativi è una
classe \emph{chiusa}, mentre quella dei qualificativi è una classe
\emph{aperta} che prevede cioè la possibilità di espandersi in maniera
indefinita.

\hypertarget{articolo}{%
\subsection{Articolo}\label{articolo}}

L'articolo è quella particella che si accompagna al nome o ad altre
parti del discorso in funzione sostantivata. In italiano esso concorda
nei tratti di Numero, Persona, Genere con il sostantivo di riferimento
\citep{grandi2010}. Le lingue del mondo non presentano tutte lo stesso
comportamento nei riguardi della posizione e/o della presenza
dell'articolo e possiamo trovare:

\begin{itemize}
\tightlist
\item
  Lingue senza articoli (\emph{ceco, slovacco})
\item
  Lingue con articoli

  \begin{itemize}
  \tightlist
  \item
    Proclitici (\emph{italiano, inglese})
  \item
    Enclitici (\emph{bulgaro, macedone})
  \end{itemize}
\end{itemize}

In una lingua come l'italiano, la presenza dell'articolo è lo
\emph{standard}, ovvero non ha una funzione specifica mentre la sua
assenza assume significato. Così, per esempio, in \textbf{italiano
standard}\footnote{Alcune varietà di italiano, quali i dialetti
  settentrionali, hanno invece gli articoli in questi contesti
  \citep{loporcaro2009}.} i nomi propri escludono l'articolo
(\emph{Marta va in città} vs.~\emph{*La Marta va in città}) così come è
esclusa la possibilità di trovare l'articolo in combinazione con il
possessivo nei nomi di famiglia (\emph{mio figlio si chiama Luigi}
vs.~\emph{* Il mio figlio si chiama Luigi}).

\begin{table}[]
\begin{tabular}{@{}lrrrrrr@{}}
\toprule
      & \multicolumn{2}{c}{Definito} & \multicolumn{2}{c}{Indefinito} & \multicolumn{2}{c}{Partitivo} \\ \midrule
      & Sing          & Plur         & Sing           & Plur          & Sing          & Plur          \\
Masch & il            & i            & un             & -             & del           & dei           \\
Masch & lo            & gli          & uno            & -             & dello         & degli         \\
Fem   & la            & le           & una            & -             & della         & delle         \\ \bottomrule
\end{tabular}
\caption{Tabella riassuntiva degli articoli in italiano}
\end{table}

\hypertarget{definito}{%
\subsubsection{Definito}\label{definito}}

L'articolo definito o \emph{determinativo} può indicare un referente
determinato, ovvero noto (\emph{Sto cercando il libro}, \emph{hai visto
la mia maglietta?}):

\hypertarget{indefinito}{%
\subsubsection{Indefinito}\label{indefinito}}

Quello indefinito o indeterminativo può essere usato per indicare un
sostantivo indefinito specifico (\emph{non trovo un libro che avevo
lasciato a casa}) oppure non specifico (\emph{per la nuova casa vorrei
trovare un inquilino simpatico}). Gli articoli indefiniti non possono
essere usati al plurale e la loro forma è la stessa del numero
\enquote{uno} (1).

\hypertarget{partitivo}{%
\subsubsection{Partitivo}\label{partitivo}}

L'articolo partitivo si usa per indicare quantità indefinite o parti di
un insieme (\emph{vorrei del pane}, \emph{sto cercando dei libri},
\emph{la maggior parte dei ragazzi pensa solo a una cosa}). Si forma
dall'unione delle forme \enquote{di} con l'articolo definito
(\emph{del}, \emph{dello}, \emph{della}, \emph{dei}, \emph{degli},
\emph{delle}).

\hypertarget{nome}{%
\subsection{Nome}\label{nome}}

\hypertarget{pronome}{%
\subsection{Pronome}\label{pronome}}

\hypertarget{verbo}{%
\subsection{Verbo}\label{verbo}}

\hypertarget{le-parti-invariabili}{%
\section{Le parti invariabili}\label{le-parti-invariabili}}

\hypertarget{avverbio}{%
\subsection{Avverbio}\label{avverbio}}

\hypertarget{congiunzione}{%
\subsection{Congiunzione}\label{congiunzione}}

\hypertarget{interiezione}{%
\subsection{Interiezione}\label{interiezione}}

\hypertarget{preposizione}{%
\subsection{Preposizione}\label{preposizione}}

\hypertarget{la-frase}{%
\chapter{La frase}\label{la-frase}}

\hypertarget{coordinazione}{%
\section{Coordinazione}\label{coordinazione}}

\hypertarget{giustapposizione}{%
\section{Giustapposizione}\label{giustapposizione}}

\hypertarget{connettori}{%
\section{Connettori}\label{connettori}}

\hypertarget{esempi}{%
\section{Esempi}\label{esempi}}

\part*{Parte II. \\\ La frase complessa}

\hypertarget{frasi-soggettive}{%
\chapter{Frasi soggettive}\label{frasi-soggettive}}

\hypertarget{funzione}{%
\section{Funzione}\label{funzione}}

\hypertarget{soggettive-esplicite}{%
\section{Soggettive esplicite}\label{soggettive-esplicite}}

\hypertarget{soggettive-implicite}{%
\section{Soggettive implicite}\label{soggettive-implicite}}

\hypertarget{frasi-oggettive}{%
\chapter{Frasi oggettive}\label{frasi-oggettive}}

\hypertarget{funzione-1}{%
\section{Funzione}\label{funzione-1}}

\hypertarget{oggettive-esplicite}{%
\section{Oggettive esplicite}\label{oggettive-esplicite}}

\hypertarget{oggettive-implicite}{%
\section{Oggettive implicite}\label{oggettive-implicite}}

\hypertarget{frasi-interrogative}{%
\chapter{Frasi Interrogative}\label{frasi-interrogative}}

\hypertarget{funzione-2}{%
\section{Funzione}\label{funzione-2}}

\hypertarget{dirette}{%
\section{Dirette}\label{dirette}}

\hypertarget{indirette}{%
\section{Indirette}\label{indirette}}

\hypertarget{esplicite}{%
\section{Esplicite}\label{esplicite}}

\hypertarget{implicite}{%
\section{Implicite}\label{implicite}}

\hypertarget{frasi-relative}{%
\chapter{Frasi Relative}\label{frasi-relative}}

\hypertarget{tipi}{%
\section{Tipi}\label{tipi}}

\hypertarget{esplicite-1}{%
\section{Esplicite}\label{esplicite-1}}

\hypertarget{implicite-1}{%
\section{Implicite}\label{implicite-1}}

\hypertarget{frasi-temporali}{%
\chapter{Frasi temporali}\label{frasi-temporali}}

\hypertarget{definizione}{%
\section{Definizione}\label{definizione}}

\hypertarget{tipi-1}{%
\section{Tipi}\label{tipi-1}}

\hypertarget{esplicite-2}{%
\section{Esplicite}\label{esplicite-2}}

\hypertarget{implicite-2}{%
\section{Implicite}\label{implicite-2}}

\hypertarget{frasi-comparative-e-modali}{%
\chapter{Frasi comparative e modali}\label{frasi-comparative-e-modali}}

\hypertarget{definizione-1}{%
\section{Definizione}\label{definizione-1}}

\hypertarget{tipi-2}{%
\section{Tipi}\label{tipi-2}}

\hypertarget{esplicite-3}{%
\section{Esplicite}\label{esplicite-3}}

\hypertarget{implicite-3}{%
\section{Implicite}\label{implicite-3}}

\hypertarget{frasi-causali-e-finali}{%
\chapter{Frasi causali e finali}\label{frasi-causali-e-finali}}

\hypertarget{definizione-2}{%
\section{Definizione}\label{definizione-2}}

\hypertarget{esplicite-4}{%
\section{Esplicite}\label{esplicite-4}}

\hypertarget{implicite-4}{%
\section{Implicite}\label{implicite-4}}

\hypertarget{frasi-consecutive-e-concessive}{%
\chapter{Frasi consecutive e
concessive}\label{frasi-consecutive-e-concessive}}

\hypertarget{definizione-3}{%
\section{Definizione}\label{definizione-3}}

\hypertarget{esplicite-5}{%
\section{Esplicite}\label{esplicite-5}}

\hypertarget{implicite-5}{%
\section{Implicite}\label{implicite-5}}

\hypertarget{frasi-codizionali}{%
\chapter{Frasi codizionali}\label{frasi-codizionali}}

\hypertarget{definizione-4}{%
\section{Definizione}\label{definizione-4}}

\hypertarget{esplicite-6}{%
\section{Esplicite}\label{esplicite-6}}

\hypertarget{implicite-6}{%
\section{Implicite}\label{implicite-6}}

\hypertarget{discorso-diretto-e-indiretto}{%
\chapter{Discorso diretto e
indiretto}\label{discorso-diretto-e-indiretto}}

\hypertarget{definizione-5}{%
\section{Definizione}\label{definizione-5}}

\hypertarget{esplicite-7}{%
\section{Esplicite}\label{esplicite-7}}

\hypertarget{implicite-7}{%
\section{Implicite}\label{implicite-7}}

\bibliography{bibliography.bib}

\end{document}
